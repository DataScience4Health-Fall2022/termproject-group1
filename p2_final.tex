%%
%% This is file `sample-acmsmall.tex',
%% generated with the docstrip utility.
%%
%% The original source files were:
%%
%% samples.dtx  (with options: `acmsmall')
%% 
%% IMPORTANT NOTICE:
%% 
%% For the copyright see the source file.
%% 
%% Any modified versions of this file must be renamed
%% with new filenames distinct from sample-acmsmall.tex.
%% 
%% For distribution of the original source see the terms
%% for copying and modification in the file samples.dtx.
%% 
%% This generated file may be distributed as long as the
%% original source files, as listed above, are part of the
%% same distribution. (The sources need not necessarily be
%% in the same archive or directory.)
%%
%% Commands for TeXCount
%TC:macro \cite [option:text,text]
%TC:macro \citep [option:text,text]
%TC:macro \citet [option:text,text]
%TC:envir table 0 1
%TC:envir table* 0 1
%TC:envir tabular [ignore] word
%TC:envir displaymath 0 word
%TC:envir math 0 word
%TC:envir comment 0 0
%%
%%
%% The first command in your LaTeX source must be the \documentclass command.
\documentclass[acmsmall]{acmart}
%% NOTE that a single column version is required for 
%% submission and peer review. This can be done by changing
%% the \doucmentclass[...]{acmart} in this template to 
%% \documentclass[manuscript,screen]{acmart}
%% 
%% To ensure 100% compatibility, please check the white list of
%% approved LaTeX packages to be used with the Master Article Template at
%% https://www.acm.org/publications/taps/whitelist-of-latex-packages 
%% before creating your document. The white list page provides 
%% information on how to submit additional LaTeX packages for 
%% review and adoption.
%% Fonts used in the template cannot be substituted; margin 
%% adjustments are not allowed.
%%
%% \BibTeX command to typeset BibTeX logo in the docs
\AtBeginDocument{%
  \providecommand\BibTeX{{%
    \normalfont B\kern-0.5em{\scshape i\kern-0.25em b}\kern-0.8em\TeX}}}

%% Rights management information.  This information is sent to you
%% when you complete the rights form.  These commands have SAMPLE
%% values in them; it is your responsibility as an author to replace
%% the commands and values with those provided to you when you
%% complete the rights form.
%%\setcopyright{acmcopyright}
%%\copyrightyear{2018}
%%\acmYear{2018}
%%\acmDOI{XXXXXXX.XXXXXXX}


%%
%% These commands are for a JOURNAL article.
%%\acmJournal{JACM}
%%\acmVolume{37}
%%\acmNumber{4}
%%\acmArticle{111}
%%\acmMonth{8}

%%
%% Submission ID.
%% Use this when submitting an article to a sponsored event. You'll
%% receive a unique submission ID from the organizers
%% of the event, and this ID should be used as the parameter to this command.
%%\acmSubmissionID{123-A56-BU3}

%%
%% For managing citations, it is recommended to use bibliography
%% files in BibTeX format.
%%
%% You can then either use BibTeX with the ACM-Reference-Format style,
%% or BibLaTeX with the acmnumeric or acmauthoryear sytles, that include
%% support for advanced citation of software artefact from the
%% biblatex-software package, also separately available on CTAN.
%%
%% Look at the sample-*-biblatex.tex files for templates showcasing
%% the biblatex styles.
%%

%%
%% The majority of ACM publications use numbered citations and
%% references.  The command \citestyle{authoryear} switches to the
%% "author year" style.
%%
%% If you are preparing content for an event
%% sponsored by ACM SIGGRAPH, you must use the "author year" style of
%% citations and references.
%% Uncommenting
%% the next command will enable that style.
%%\citestyle{acmauthoryear}

%%
%% end of the preamble, start of the body of the document source.
\begin{document}

%%
%% The "title" command has an optional parameter,
%% allowing the author to define a "short title" to be used in page headers.
\title{Biomarker feature selection for detecting Obstructive Sleep Apnea events
}

%%
%% The "author" command and its associated commands are used to define
%% the authors and their affiliations.
%% Of note is the shared affiliation of the first two authors, and the
%% "authornote" and "authornotemark" commands
%% used to denote shared contribution to the research.
\author{Piper Stacey}
\email{ piper.f.stacey.23@dartmouth.edu}
\affiliation{%
  \institution{Dartmouth College}
}
\author{Vafa Batool}
\email{ vafa.batool.gr@dartmouth.edu}
\affiliation{%
  \institution{Dartmouth College}
}
\author{John Berry}
\email{john.n.berry.gr@dartmouth.edu}
\affiliation{%
  \institution{Dartmouth College}
}
\author{Shitong Cheng}
\email{shitong.cheng.gr@dartmouth.edu}
\affiliation{%
  \institution{Dartmouth College}
}





%%
%% By default, the full list of authors will be used in the page
%% headers. Often, this list is too long, and will overlap
%% other information printed in the page headers. This command allows
%% the author to define a more concise list
%% of authors' names for this purpose.


%%
%% The abstract is a short summary of the work to be presented in the
%% article.
\begin{abstract}
The abstract is here!
\end{abstract}

%%
%% The code below is generated by the tool at http://dl.acm.org/ccs.cfm.
%% Please copy and paste the code instead of the example below.
%%


%%
%% Keywords. The author(s) should pick words that accurately describe
%% the work being presented. Separate the keywords with commas.
\keywords{datasets, sleep apnea, machine learning, text tagging}

\maketitle

%%
%% This command processes the author and affiliation and title
%% information and builds the first part of the formatted document.


\section{Introduction}

Obstructive sleep apnea is a form of disordered breathing while sleeping that occurs particularly due to an obstruction in the upper airway \cite{davis2013early}. An estimated 80\% to 90\% of all obstructive sleep apnea cases go undiagnosed \cite{silverberg2002treating}. The problem with obstructive sleep apnea syndrome is that patients do not get enough high quality sleep. Lower quality sleep leads to accidents while higher quality sleep promotes attentiveness and reduces blood pressure. Furthermore, there is a known association between obstructive sleep apnea and hypertension, metabolic syndrome, diabetes, heart failure, coronary artery disease, arrhythmias, stroke, pulmonary hypertension, neurocognitive and mood disorders \cite{mannarino2012obstructive}. Specifically in stroke patients, untreated sleep apnea is associated with major risk factors for another stroke. The prevalence of sleep apnea is 50\%-70\% in stroke patients. Earlier diagnoses and treatments of sleep apnea in stroke patients could improve recovery from stroke and reduce the likelihood of another stroke in the future \cite{davis2013early}. 

There has been a lot of work studying ways to leverage data science to detect sleep apnea but they have focused on only a few datapoints across the range of what is commonly collected. There has not yet been a study that seeks to determine which of those commonly focused on result in improved accuracy when automatically detecting sleep apnea events.


\subsection{Objective and Search Plan}

We have many dimensions of data, such as HR, SpO2, RR, etc, not every type of data is sensitive to the detection of apnea, or they are well suited as data for detecting apnea. It is also known that the choice of model or algorithm will determine the accuracy of the final result to some extent. We strive to select the optimal combination of subsets of the data and algorithmic, such that the detection of apnea is as accurate and sensitive as possible. 

Previous work has been done based upon a variety of learning algorithms though nothing has been published comparing the efficacy of one algorithm over the other in dealing with this problem space. We will select several algorithms, likely some subset of K-means clustering, SVMs, Reinforcement learning, Random forests, and others, upon which to base our modelling. With each of these algorithms we will train models on subsets of the features included in our data set. For example, one model uses HRV, RR, SpO2 and some Polysomnograpy readings while another model uses just HRV and RR. etc. From here, the goal is to determine which combination of models and features produces the greatest performance in respect to accuracy and sensitivity. (Type I and Type II errors).


\subsection{Contribution}
There has not been a concerted effort to determine if combinations of these data points and algorithms can perform better than the models trained on the data points individually. Using the features provided in \cite{bernardini2022osasud} such as the perfusion index, respiratory rate, detected snores, and others.This paper presents a comparison between modelling methodologies in conjunction with feature set selections that when combined provide the best predictive capabilities of OSA events in patients. OSA has been associated with many negative health outcomes though early detection and management can reduce these. Finding a model that best detects these events can lead to improved patient treatment plans.

\section{Related Work}
Previous works in the space of Obstructive Sleep Apnea (OSA) have explored its impact on the quality of life of individual. It includes how it may negatively impact the posture and gait and have attributed it as among a leading cause of falls in older individuals. They have focused on treating OSA and measuring its potential effect, if any, on injuries sustained due to falls \cite{stevens2020impact}. Muraki et. al have also contributed towards understanding how OSA may be correlated with other health conditions like type II diabetes. They help uncover how continuous positive airway pressure (CPAP), the main treatment for OSA, may help improve insulin resistance as a way for better treating diabetes patients \cite{muraki2018sleep}. Some past studies have helped uncover how OSA may be contributing to comorbidities like hypertension. Based on the meta-analysis of PubMed and Embase databases, it was found that OSA is linked to increased hypertension, especially in male Caucasian individuals \cite{hou2018association}. The relationship between OSA and strokes has also been studied previously. Dyken et. al found a close causal relationship between the two where the events of snoring and recorded instances of apnea before a stroke have revealed that untreated OSA may lead to strokes \cite{dyken2009obstructive,silverberg2002treating}. Studies that have focused on stroke patients with apnea have also investigated through a longitudinal study the outcome of OSA on their life. It was found that after a year the survival rate of apnea patients, especially men, was significantly lower than the patients without apnea problems \cite{kojic2021does}.
 
 In previous studies, machine learning techniques have proven to be useful with regard to picking bio-markers that are able to predict a patient's future condition \cite{zhang2021machine}. Specifically, researchers have used single-channel nasal pressure airflow signals to diagnose OSA \cite{haidar2017sleep}. Other study diagnosed OSA based on hyperparameters such as blood reports, demographics, physical measurements, comorbidities, and sleep habits \cite{ramesh2021towards}. Others have used ECG spectrograms to diagnose OSA \cite{lin2021sleep,erdenebayar2019deep,tison2019automated} while some have chosen to focus on using respiratory rate and oxygen saturation \cite{ravelo2015oxygen}. Furthermore, studies have used machine learning and support vector machine predictions on polysomnography data to predict OSA \cite{palotti2019benchmark,huang2020support}. Finally, in the past there have been attempts at identifying genes associated with OSA and CPAP to improve the diagnostic accuracy \cite{bernardini2022osasud}. 
 
 % Gap in the previous works
 As such, previous works have focused largely on understanding the link between OSA and other health condition, how it can be treated more effectively and its detection based on genes involved. There has been no attempt at using the biomarkers available such as oxygen levels overnight or respiratory rate to predict the onset of apnea events. This is where the novelty of our work would step in. 



\bibliographystyle{acm}
\bibliography{ref}

\end{document}


\endinput
%%
%% End of file `sample-acmsmall.tex'.
